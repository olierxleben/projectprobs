%
%
\documentclass[12pt]{scrartcl}

% own geometry
\usepackage[a4paper, left=3cm, right=3cm]{geometry}

\usepackage[ngerman]{babel} 
\usepackage[utf8]{inputenc} 
\usepackage[T1]{fontenc}
\usepackage{graphicx}
\usepackage{color}
\usepackage{xcolor}
\usepackage{jurabib}
\usepackage{hyperref}
\usepackage{here} % picture positioning

\include{lib/jurabib}
\bibliographystyle{jurabib}

% setup of source code listings
\usepackage{listings}
%\usepackage{courier}
\usepackage{caption}

\DeclareCaptionFont{white}{\color{white}}
\DeclareCaptionFormat{listing}{\colorbox{gray}{\parbox{\textwidth}{#1#2#3}}}
\captionsetup[lstlisting]{format=listing,labelfont=white,textfont=white}

% layout the box
%\DeclareCaptionFormat{listing}{\colorbox[rgb]{0.43, 0.35, 0.35 {\parbox{\textwidth}{\hspace{15pt}#1#2#3}}}

% Headings
\usepackage{fancyhdr}
\renewcommand{\footrulewidth}{1pt}
%\fancyhead[R]{\colorbox{blue!20}{ Oliver Erxleben}}
\fancyfoot[C]{}
\fancyfoot[R]{\thepage}

% Document begins now
\begin{document}

\author{
	Oliver Erxleben \small(\href{mailto:oliver.erxleben@hs-osnabrueck.de}{oliver.erxleben@hs-osnabrueck.de})\\ \\%
	%
	Hochschule Osnabr"uck \\%
	Ingenieurswissenschaften und Informatik \\%
	Informatik - Mobile und Verteilte Anwendungen }

\title{\includegraphics[scale=0.75,keepaspectratio]{img/hs_os.png}\linebreak \linebreak Probleme im Projektmanagement und Führungstipps}

\maketitle
\thispagestyle{empty}
\pagebreak
\thispagestyle{empty}
\tableofcontents
\listoffigures
\thispagestyle{empty}
\pagebreak
\thispagestyle{empty}
\begin{abstract}
\textbf{Zusammenfassung}\\
Die vorliegende Arbeit wurde mit LaTex verfasst und ist eine Arbeit von Oliver Erxleben für das Modul \textit{Projektmanagement und Führungstheorien} aus dem Master-Studiengang \textit{Informatik - Verteilte und Mobile Anwendungen} der \textit{Hochschule Osnabrück / University of Applied Sciences}.\\
\\
Das Thema der Arbeit lautet \textbf{Probleme im Projektmanagement und Führungstipps}. Es wird auf typische Problemstellen eines Projekts eingegangen, mit der das Projektmanagement konfrontiert wird. Die einzelnen Probleme werden dabei aus Sicht verschiedener Projektbeteiligter dargestellt.\\
Die Arbeit ist in mehrere Abschnitte aufgeteilt. In der Einleitung wird nach der Motivation für Probleme im Projektmanagement auf Eckdaten und Begriffsefinitionen eingegangen.\\ % TODO: überarbeiten an neues Inhaltsverzeichnis
Die darauffolgenden Abschnitte sind grob in die Phasen eines Projekts aufgeteilt, in der Probleme dieser Phase geschildert und analysiert werden. Nach jedem Abschnitt findet sich eine kurze pragmatische Zusammenfassung in der auch Tipps zur Problembewältigung gegeben werden.\\
Im letzten Abschnitt werden die Erkenntnisse resümiert und ein Ausblick gegeben, wie Problemen in Zukunft vorgebeugt werden können.\\
\\
Es sei erwähnt dass die Arbeit auf IT-Projekte abzielt. Besonders werden IT-Projekte, bzw. Projekte in der Software-Entwicklung betrachtet. Projektmanagement aus anderen Branchen, wie Bauingenieurswesen oder Projektmanagement in der Pflege, wird in dieser Arbeit nicht betrachtet. 
\end{abstract}

\pagebreak
% set new page style

\pagestyle{fancy}
\setcounter{page}{1} 

\section{Einleitung}

Projekte scheitern! Projekte scheitern in der IT häufiger als sie Erfolg haben. Und das obgleich die Disziplin des \textit{''Projektmanagaments''} keine neue Erfindung ist und diese sich ständig weiter entwickelt und verstärkte Aufmerksamkeit in Unternehmen erfährt. So ist in den letzten Jahren zu verzeichnen das Projektportfolios und Projektauswahl mit der Unternehmens- und Geschäftsfeldstrategie abgestimmt werden. Fakt ist das Strategien, Produkte, Dienstleistungen oder Innovationen durch Projekte realisiert werden. Dies wird in Unternehmen immer mehr thematisiert.\\
Diese Tendenz ist zwar gut, aber aufgrund der harten Fakten in Studien trotz allem scheinbar nicht genug. Diese Studien belegen das 46 \%\footnote{''Chaos-Report'' der Standish Group 2006} der IT-Projekte in Deutschland Wünsche und Anforderungen von Auftraggebern nur teilweise oder gar nicht erfüllt haben. Hinzukommt das 20 \% von IT-Projekten abgebrochen werden und nur 16 \% der IT-Projekte können als \textit{erfolgreich} eingestuft werden. Warum also scheitern so viele Projekte in der IT und welche Probleme und Problemsituationen gibt esin einem Projekt? Sind die Fachkompetenzen zu gering? Sind die Ziele eines Projekts zu hoch gegriffen? Ist die Führungsebene eines Unternehmens schuld? Oder arbeiten alle enfach nur zu wenig?\\
Ähnliche Ergebnisse zum Erfolg, bzw. Misserfolg eines Projekts liefert auch eine Studie der GPM\footnote{deutsche Gesellschaft für Projektmanagement e.V.}, in der auch mögliche Probleme in IT-Projekten und Ursachen für das Scheitern von Projekten versucht wird herauszufinden (Details siehe Abschnitt \ref{studies_gpm}).\\ 
Jedes Projekt, egal ob erfolgreich oder nicht, hat oder hatte Probleme zu verzeichnen. Seien es nun Probleme, wie z.B., Interessenskonflikte zwischen Projektbeteiligten und -mitarbeitern, sich ändernde Anforderungen oder das Problem das Projekt im Projektverlauf auf Kurs zu halten. \\
Es ist die Kunst der Führung und des Projektmanagements, sowie der Führungsebene sich bildende oder aftretende Probleme  und kritische Pässe innerhalb eines Projekts frühzeitig zu erkennen, vorzubeugen oder schnell zu lösen. Dies bedarf allerdings Probleme eines Projekts und damit Probleme des Projektmanagements zu kennen und durch geeignete Operationen in der Führung diese Probleme zu lösen.

\pagebreak
% TODO: überarbeiten
\subsection{Aufbau und Ablauf der Arbeit}
Die vorliegende Arbeit gliedert sich in vier Teile: Einleitung, Projektplanung, Projektdurchführung, Projektabschluss und Fazit, wobei die Kapitel zwei bis vier den Hauptteil der Arbeit ausmachen. 
\\ \\
Im einleitenden Teil wird in das Thema der Arbeit eingeführt und es werden wichtige Begriffe für die restliche Arbeit definiert und abgegrenzt.  
\\ \\
Der zweite Teil der Arbeit beschäftigt sich mit der aktuellen Lage in der Projektmanagementkultur und wertet Studien aus die zeigen welche Faktoren maßsgeblich für erfolgreiche Projekte sind und welche Probleme das Projektmanagement, bzw. das Projekt dabei erfahren kann. 
\\ \\
Die darauffolgenden Abschnitte sind die Quintessenz aus dem Ergebnis des zweiten Abschnitts. Folglich werden Probleme geschildert die bei den größten Erfolgsfaktoren für Projekte entstehen können. Die Kommunikationsprobleme, Projektziele, Protjektleiterpositionen, Ressourcenverteilung und Projektmanagement-Prozesse vorhanden sind. 
\\ \\
Im Teil Fazit werden Erkenntnisse aus dem Hauptteil zusammengefast und bewertet.
\\ \\
Nach jedem Abschnitt aus dem Hauptteil werden Probleme zusammengefasst und als Gegenüberstellung mit Lösungsanseätzen gelistet. 
\subsection{Begriffsdefinitionen}

Im Folgenden werden zentrale Begriffe für die Arbeit definiert. 

\subsubsection{Projekt}

Ein Projekt verfolgt immer ein Ziel, also eine bestimmte \textit{Zielvorgabe}. Zum Beispiel die Umsetzung eines Auftragsarbeit, oder die Erstellung einer Smartphone-Anwendung. Die Zielvorgabe besteht aus mehreren Einzelaufgaben und ist \\
\\
Ein Projekt nach Aussage der DIN-Begriffsnorm 69 901 allgemein defniniert durch \texttt{''ein Vorhaben, das im Wesentlichen durch die Einmaligkeit der Bedingung in ihrer Gesamheit gekennzeichnet ist, wie z.B. 
\begin{itemize}
    \item{Zielvorgabe}
    \item{zeitliche, finanzielle, personelle und andere Begrenzungen}
    \item{Abgrenzung gegenüber anderen Vorhaben}
    \item{projektspezifische Organisation}
\end{itemize}''}
gekennzeichnet. Im weiteren Vergleich mit \cite{proj_zum_erfolg_fuehren} (Kapitel 2, Seite 19 ff.) wird die in der DIN-Begriffsnorm eingetragenen Merkmale noch um die Beteiligung von Menschen, Arbeitsgruppen und Institutionen hinzugefügt und die Existenz eines \textit{Ein-Personen-Projekts} ausgeschlossen. Mit beiden Ergänzungen und der DIN-Definition wird in den nachfolgenden Kapiteln ein Projekt verstanden. 

\subsubsection{Phasen, Meilensteine und Aktivitäten}

Unter \texttt{Projektphasen} werden Ereignisse in einem Phasenmodell\footnote{auch bez. als Phasenmodell, Vorgehensmodell, Prozessmodell} verstanden, die zeitleich aufeinander folgen. So kann das Testen einer Software beispielsweise nicht vor dessen Planungsphase stattfinden. Das Phasenmodell dient als Hilfsmittel im Projektmanagement. \\
\\
Ein \texttt{Meilenstein} ist im Vergleich zur Projektphase (nach DIN 69 900) \texttt{ein Ereignis besonderer Bedeutung}. Meist erhält ein Meilenstein einen geplanten Termin und Inhalte die zu diesem Termin vom Projekt umgesetzt sein müssen. \\
\\
Zusätzlich zu Phasen und Meilensteine können auch \texttt{Aktivitäten} definiert werden. Sie beschreiben \textit{was} in den verschiedenen Projektabschnitten getan werden muss um ein Teilergebnis erzielen zu können.  

\subsubsection{Projektmanagement}
Die DIN definiert Projektmanagement als \texttt{die Gesamtheit von Führungsaufgaben, -organistion, -techniken und -mittel für die Abweicklung eines Projekts}.\\
\\
Von der Disziplin \texttt{Projektmanagement} kann, wenn auch schon immer Vorhaben durchgeführt wurden, seit den 1950ern geredet werden. Geprägt wurde das Projektmanagement stark vom Militär und der Luft- und Weltraumindustrie der USA. Dort waren starke Termin- und Kostenüberschreitungen fatal für die Auftraggeber. Gerade die Luft- und Raumfahrtindustrie trug maßgeblich zu den anfänglichen Techniken im Projektmanagement bei\footnote{Vgl. \cite{proj_zum_erfolg_fuehren}, Die Entwicklung der Disziplin Projektmanagement, Seite 23 f.}. 
\\
Es kann aber auch als \textit{Führungskonzept} verstanden werden (Vgl. \cite{scriptPM}, Seite 7), bei dem die Aufgaben und Methoden des Projektmanagements auf die strategischen Zielen und der Entwicklung des Unternehmens verknüpft werden müssen. 

\subsubsection{Projektleitung}
% TODO: \cite proj_zum_erfolg_fuehren ?!
Als Projektleitung wird die omni-verantwortliche Person für ein Projekt bezeichnet. \cite{proj_zum_erfolg_fuehren} bezeichnet den Projektleiter als \texttt{Unternehmer auf Zeit im Unternehmen}. \\
\\
Der Projektleiter hat die Verantwortung über den Erfolg oder Misserfolg seines Projekts. Es ist \textit{sein} Projekt. Er ist verantwortlich für die Realisierung der definierten Projektziele, Termineinhaltung, Kosten- und Qualitätsgarantie und der Koordination der Projektbeteiligten. 

\subsubsection{Projektbeteiligte}
% TODO: weitere Projektbeteiligte + Beschreibung
Neben dem Projektmanagement und dem Projektleiter sind auch noch weitere Personen und Gruppen von Personen an einem Projekt beteiligt:
\begin{description}
    \item[Auftraggeber]
    Auch bekannt als 
    \item[Projektmitarbeiter]
\end{description}

\pagebreak
\section{Projekte scheitern}
\label{projekte_scheitern}
Dieser Abschnitt wertet verfügbare Studien aus, die für die Erörterung von PM-Problemen maßgeblich sind, fasst diese zusammen und gibt einen Überblick über die Situation (dem \textit{warum} Projekte scheitern) nach Aussage dieser Studien. 

\subsection{Aktuelle Lage}
% TODO: Studie Gesamt-Studie GPM 2007 bis 2009 
Wie gut werden Projekte gemanagt? Studien unterschiedlicher Gesellschaften, Unternehmen oder Redaktionen belegen das zuviele Projekte abgebrochen werden oder erheblich mehr Zeit in Anspruch nehmen, als ursprünglich geplant. So besagt eine Studie der Zeitschrift \textit{Computerwoche} zusammen mit dem Institut für Betriebswirtschaftslehre der TU München, dass gerade nur 43 Prozent der IT-Projekte in den letzten drei Jahren erfolgreich waren. 48 Prozent benötigten mehr Zeit, kosteten mehr oder hatten ein nicht geplantes Ergebnis. Ähnliche Ergebnisse liefert auch die Standish Group. \\
\\
% TODO: Kuchengrafik als Abbildung
Es stellt sich also aktuell die Frage, was beeinflußt den Projekterfolg? Welche Probleme können dort auftreten? Wie könnte man Probleme lösen?

\subsection{Studie der GPM}
\label{studies_gpm}

% TODO: footnote verfassen
Die deutsche Gesellschaft für Projektmanagement e.V., gegründet 1979, hat sich als \textit{der} Berufsverband für Projektmanagement etabliert. Ihr gehören ca. 250 Unternehmen und  über 4500 Mitglieder an\footnote{Stand 31.12.2008, Vgl. siehe \cite{proj_zum_erfolg_fuehren}, Kapitel 24 Seite 311}. Neben Publizierungen, wie die Zeitschrift ''Der Projektmanager'', bietet die GPM auch Zertifizierungen oder Weiterbildungsseminare an und betreibt Studien. \\
Eine dieser Studien, die Projektmanagementstudie aus dem Jahr 2008, ist für diese Ausarbeitung besonders nützlich: \texttt{Erfolg und Scheitern im Projektmanagement}. Sie ist in Kooperation mit der PA Consulting Group\footnote{} entstanden und beschreibt nicht nur die aktuellen Probleme in Projekten und dem Projektmanagement, sondern versucht auch Gründe für das Scheitern, wie auch für den Erfolg von Projekten herauszufinden. 

\subsubsection{Erhobene Daten}
\label{erhobene_daten}
Die Studie erhebt zweierlei Arten von Daten. Zum einen wurden Fragen zur Projektkultur gestellt (allgemeiner Natur), zum anderen spezielle Fragen zu erfolgreichen und gescheiterten Projekten. Die Abbildung \ref{fragen_gpm_studie} zeigt exemplarische Fragen zur Projektkultur und zu den speziellen Projekten. 

\begin{figure}[H]
	\begin{center}
		\includegraphics[width=0.9\textwidth]{img/fragen_gpm_studie}
		\caption{Fragen der GPM-Studie}
		\label{fragen_gpm_studie}	
	\end{center}
\end{figure}
\ 
\\
Für die vorliegende Ausarbeitung ist allerdings wesentlich interessanter was die Gründe für Erfolg oder Misserfolg von einzelnen Projekten sei. Dazu hat die GPM die teilnehmenden Unternehmen zu erfolgreichen und gescheiterten Projekten befragt. Für die Auswertung der Einzelprojekte wurde folgende Methodik angewandt\footnote{Vgl. \cite{GPM_Studie_2008, Seite 9}}:
\begin{itemize}
    \item{Fragen für erfolgreiche und gescheiterte Projekte sind identisch}
    \item{Für jede Frage wurde aus jeder Antwort ein Durchschnittswert ermittelt; jeweils für ''erfolgreiches Projekt'' und für ''gescheitertes Projekt''}
    \item{Durchschnittswert liegt zwischen 1 (''trifft überhaupt nicht zu'') und 5 (''Trifft voll und ganz zu'').}
    \item{Durchschnittswerte einer Frage wurden den beiden Projektkategorien (gescheitert, erfolgreich) gegenübergestellt}
    \item{Differenz soll Aufschluss über Faktoren für Projekterfolg liefern}
\end{itemize}
\
\\
Details zu den Ergebnissen finden sich in den Abschnitten \ref{ergebnis_pmk} (Projektmanagement-Kultur) und \ref{ergebnis_einzel} (Einzelprojekte)

\subsubsection{Teilnehmer}

Neben der Frage \textit{welche} Daten erhoben werden, muss auch geklärt werden, \texttt{wer} an der Studie teilgenommen hat. Dies waren 79 Unternehmen, überwiegend Organisationen mit mehr als 1000 Mitarbeitern und auch überwiegend Jahresumsätze von mehr als 1 Mrd. EUR (\textit{Vgl. siehe \cite{GPM_Studie_2008}, Seite 4}).\\
Es wurde versucht eine hohe Branchenvielfalt zu erreichen. Die Abbildung \ref{teilnehmer_gpm_studie} zeigt die prozentuale Teilnahme verschiedenster Branchen am Markt. Zu erwähnen sei noch, dass es einen hohen Anteil an von Unternehmen teilgenommen haben, die schon an früheren Studien der GPM oder PA-Group teilgenommen haben. 

\begin{figure}[H]
	\begin{center}
		\includegraphics[width=0.9\textwidth]{img/teilnehmer_gpm_studie}
		\caption{Teilnehmer der GPM-Studie}
		\label{teilnehmer_gpm_studie}	
	\end{center}
\end{figure}

\subsubsection{Ergebnis der Projektmanagement-Kultur}
\label{ergebnis_pmk}
Das Ergebnis der Befragung der Projektmanagement-Kultur ist in Abbildung \ref{erfgebnis_gpm_studie_pmk} zu sehen. Fragen konnten nach einem Notensystem beantwortet werden, wobei 1 für ''Trifft gar nicht zu'' und 5 für ''Trifft voll und ganz zu'' einzusetzen sei. \\
Als \texttt{schwach} werden diejenigen Fragen bewertert die weniger als 3,5 in ihrem Durchschnittswert erzielt haben. Hier herrscht Optimierungsbedarf. 

\begin{figure}[H]
	\begin{center}
		\includegraphics[width=1.0\textwidth]{img/ergebnis_gpm_studie_kultur}
		\caption{Ergebnis der GPM-Studie der Projektmanagement-Kultur}
		\label{erfgebnis_gpm_studie_pmk}	
	\end{center}
\end{figure}

%TODO: erweitern
Wie in Abbildung \ref{erfgebnis_gpm_studie_pmk} zu sehen ist werden zu wenig Boni für erfolgreiche Projektarbeit angeboten und auch zu wenig Karrierechancen als Projektleiter werden eingeräumt. Es fehlt also überwiegend ein Anreizsystem.\\
\\ 
Projektorganisation gleicht der Linienorganisation, nur zu wenig Erfahrung aus früheren Projekten werden dokumentiert und in anderen Projekten genutzt.

\subsubsection{Ergebnis Einzelprojekte}
\label{ergebnis_einzel}

Wie im Abschnitt \ref{erhobene_daten} beschrieben, galt es durch verschiedene Fragen an Projektbeteiligte von gescheiterten und erfolgreichen Projekten und Gegenüberstellungen dieser Fragen bestimmte Erfolgsfaktoren für Projekte zu ermitteln. Die Abbildung \ref{ergebnis_gpm_erfolgsfaktoren} zeigt die Top-10 der Gewichtungen der Antworten mit den größten Differenzen. Das rote Polygon zeigt die Gewichtung der gescheiterten Projekte und das blaue Polygon zeigt die Gewichtung der erfolgreichen Projekte. 

% TODO: Bild Erfolgsfaktoren, breite u. Höhe justieren
\begin{figure}[H]
	\begin{center}
		\includegraphics[width=1.1\textwidth]{img/ergebnis_erfolgsfaktoren}
		\caption{Die Erfolgsfaktoren in gescheiterten und erfolgreichen Projekten}
		\label{ergebnis_gpm_erfolgsfaktoren}	
	\end{center}
\end{figure}
\ \\
Die in Abbildung \ref{ergebnis_gpm_erfolgsfaktoren} zu sehenden Antworten wurden von der GPM in sechs Cluster gruppiert. Dort bestätigen sich die bisherigen Ergebnisse. Kommunikation, Zielvorgaben und Projektleiterposition sind die häufigsten Faktoren,die  den Erfolg eines Projekts beeinflussen. Die Abbildung \ref{ergebnis_gpm_gruppiert} zeigt die sechs Cluster nach der Gruppierung der Antworten. 

\begin{figure}[H]
	\begin{center}
		\includegraphics[width=0.9\textwidth]{img/ergebnis_gpm_gruppiert}
		\caption{Gruppierung der Ergebnisse}
		\label{ergebnis_gpm_gruppiert}	
	\end{center}
\end{figure}


\subsection{Auswertung und Zusammenfassung}

An dieser Stelle soll die Studie einmal kurz zusammengefasst werden und anschließend sollen beide Teilergebnisse abgegrenzt werden. \\
\\
% TODO noch weiter ausschmücken
Bei der Erhebung der Fragen zur Projektmanagementkultur kam das Ergebnis zustande, dass gerade die Anreize fehlen würden um sich als Projektleiter zu engagieren. \\
Das Ergebnis zur Befragung von Einzelprojekten gab Aufschluss das Kommunikation, Zielvorgaben und die Position des Projektleiters entscheidend seien.
% TODO: Vergleich mit weiteren Studien

\pagebreak
\section{Kommunikationsprobleme}
\label{kommunikationsprobleme}
%TODO: Zitate richtig anzeigen
\begin{quote}
\colorbox{blue!5}{\textbf{Der Widerspruch zwischen dem, was gesagt wird, und dem, was gemeint ist, ist sehr groß. Man muß ihn herausfinden.}}\footnote{Friedrich Eberling (Vorstandsvorsitzender Braun und Brunnen AG)}
\end{quote}
\ \\
Aus der \cite{GPM_Studie_2008} (siehe Abschnitt \ref{projekte_scheitern}) geht hervor, dass Kommunikation am meisten zum Projekterfolg beiträgt. Dieser Abschnitt widmet sich den Problemen der Kommunikation in Projekten. 

\subsection{Was ist Kommunikation?}
% TODO: Begriff definieren

%TODO: schreiben
\subsection{Fallbeispiel 1: Sprachbarrieren}

%TODO: schreiben
\subsection{Fallbeispiel 2: Zuviel und zu wenig Kommunikation}

%TODO: schreiben

\cite{profPM}
\cite{scriptPM}
\cite{chaosReportCriteria}

\pagebreak
\framebox{
	\colorbox{red!20}{
		\parbox{0.44\textwidth}{
Dolore te feugait nulla facilisi nam liber tempor cum soluta nobis eleifend option. Amet consectetuer adipiscing elit sed diam nonummy nibh euismod tincidunt ut laoreet. In iis qui facit eorum; claritatem Investigationes demonstraverunt lectores legere me. In vulputate velit esse molestie consequat vel illum dolore. Wisi enim ad minim, veniam quis nostrud exerci. Facer possim assum typi non habent claritatem insitam est usus legentis lius quod.
		}
	}
}
\framebox{
	\colorbox{green!20}{
		\parbox{0.44\textwidth}{
Dolore te feugait nulla facilisi nam liber tempor cum soluta nobis eleifend option. Amet consectetuer adipiscing elit sed diam nonummy nibh euismod tincidunt ut laoreet. In iis qui facit eorum; claritatem Investigationes demonstraverunt lectores legere me. In vulputate velit esse molestie consequat vel illum dolore. Wisi enim ad minim, veniam quis nostrud exerci. Facer possim assum typi non habent claritatem insitam est usus legentis lius quod.
		}
	}
}

\pagebreak
\section{Projektziele}
\begin{quote}
\colorbox{blue!5}{\textbf{Fleiß für die falschen Ziele ist noch schädlicher als Faulheit für die richtigen.}}\footnote{Peter Bamm (1897 - 1975, dt. Arzt u. Schriftsteller)}
\end{quote}
\ \\
Ein weiterer Faktor für den Projekterfolg ist nach den Studien aus dem Abschnitt \ref{projekte_scheitern} die Zieldefinitionen eines Projekts. Dieser Abschnitt beschäftigt sich mit Problemen in der Zielsetzung eines Projekts. 

\subsection{Zielfindung und -darstellung}
% TODO: Einleitender Satz mit vage Projektziele
Die Zielfindung eines Projekts kann im wesentlichen auf zwei Arten geschehen:
\begin{itemize}
    \item{intuitives Brainstorming\footnote{es werden mehr oder weniger Zielvorstellungen intuitiv gesammelt.}}
    \item{systematischen Ableiten\footnote{Aus einem Oberziel wird versucht systematisch weitere Ziele abzuleiten.}}
\end{itemize}
\ \\
Welche Art gewählt wird, hängt von den Beteiligten, dessen Erfahrungswerte und den Neuheitsgrad des Projekts ab. Es sei auch gesagt das die Präzisierung von Zielen gerade in der Anfangsphase eines Projekts höchstwahrscheinlich mehrmals geschieht, da die Details über die Ziele erst mit dem Beginn der Realisierung deutlich werden. \\
\\
Hier könnte ein erstes Problem vorhanden sein: Wer ist beteiligt an der Zielfindung eines Projekts? An der Zieldefinitionsphase sollten auf jeden Fall das Projektmanagement, der Projektleiter, Projektmitarbeiter und, sehr wichtig, der Auftraggeber beteiligt sein. Im Zweifelsfalle sollten alle Projektbeteiligten am Zielfindungspozess mitarbeiten. Nur so kann eine engere Zusammenarbeit gewährleistet werden.\\
\\
% TODO: Darstellungsprobleme weiter schreiben
Ein weiteres Problem das sich aus der Zielfindung ergibt ist die Zieldarstellung. Sind die Projektziele verständlich visualisiert (Kann jeder Projektbeteiligte es verstehen oder kann es unterschiedlich gedeutet werden)? Welche Prioritäten liegen auf den einzelnen Projektzielen? \\
\\
% TODO: Lösungsansätze (Workshops, Kick-Off-Meeting, Überprüfung bei Meilensteinen)
Zur Lösung der Probleme der Zielfindung und -darstellung haben sich verschiedene Ansätze bewärt. 
\subsection{Zielformulierung}
Hauptursache für Probleme in einem Projekt sind die schlecht formulierten Projektziele. Bevor auf die eigentlichen Probleme eingegangen wird, sollen unterschiedliche Arten von Projektzielen abgregrenzt werden
% TODO: Projekte zum Erfolg führen Seite 86 ff. Beschreibungen adden
\begin{description}
    \item[Ergebnisziele]
    
    \item[Prozessziele]
    \item[Kundenziele]
    \item[Mitarbeiterziele]
\end{description}

Schlecht formulierte Projektziele 

\subsection{WISCY-Syndrom}
Ein Problem, mit dem ein Projektmanagement konfrontiert werden kann ist das, in den USA benannte, \texttt{Why isn`t Sam coding yet?}-Syndrom. Frei übersetzt: Warum ist der Bursche noch nicht am programmieren? (Vgl. \cite{proj_zum_erfolg_fuehren}, Kaptiel 7 Seite 85) \\
Zu oft möchte man als Team schnell mit der Realisierung beginnen, oder der Projektleiter will schnell erste Ergebnisse sehen. Sehr oft wird die Zieldefinitionsphase dabei zu schnell beendet. Das Ergebnis sind meist unklare Ziele und schlechte Zielformulierungen, die im späteren Projektverlauf dann teuer korrigiert werden müssen.
% TODO: wie kann das gelöst werden? 

\pagebreak
\section{Rollenverteilung und Projektleiter}

\pagebreak
\section{Mangelnde Ressourcenverteilung}

\pagebreak
\section{Fazit}

\pagebreak
\addcontentsline{toc}{section}{Literaturverzeichnis} % Eintrag ins Inhaltsverzeichnis
\bibliography{bib/bibliography}
\appendix

\end{document}
