%
% TODO: 
% Abbildungsreferenzen werden nicht gesetzt!!!!
% Anhänge referenzieren
%
\documentclass[11pt]{scrartcl}

% own geometry
\usepackage[a4paper, left=2cm, right=2cm]{geometry}

\usepackage[ngerman]{babel} 
\usepackage[utf8]{inputenc} 
\usepackage[T1]{fontenc}
\usepackage{graphicx}
\usepackage{color}
\usepackage{xcolor}
\usepackage{picture}

\usepackage[numbers,round]{natbib}

% setup of source code listings
\usepackage{listings}
\usepackage{courier}
\usepackage{caption}

\usepackage{fancyhdr}
\fancyfoot[C]{}
\fancyfoot[R]{Seite \thepage} %Seitennummer
\renewcommand{\footrulewidth}{0.4pt} %untere Trennlinie



%-------------------
% Begin Document Main
%-------------------
\begin{document}


\author{\\ \\ Oliver Erxleben \\ oliver.erxleben@hs-osnabrueck.de 
\\ \\ Hochschule Osnabr"uck \\
Ingenieurswissenschaften und Informatik \\ Informatik - Mobile und Verteilte Anwendungen
\\ \\ Modul: Projektmanagement und Führungstheorien \\ Wintersermester 2012/13 }

\title{\includegraphics[scale=0.75,keepaspectratio]{img/hs_os.png}\linebreak \linebreak
\\ \ Probleme im Projektmanagement und Führungstipps}

\maketitle
\thispagestyle{empty}

\pagebreak

\tableofcontents

\thispagestyle{empty}
\pagebreak

\pagestyle{fancy}
% hoooray page 1! begins 
\setcounter{page}{1} 
\section{Einleitung}
Studien !!!!! belegen, dass weitaus mehr Projekte in Unternehmen scheitern, als das sie Erfolg haben. Woran liegt das? Zum Einen 
\\
Das Ziel dieser Arbeit ist es typische Probleme im Projektmanagement und im Projektverlauf zu verdeutlichen, sowie Tipps und Ideen zu vermitteln eben jene Probleme zu beseitigen - oder besser - vorzubeugen.

\cite{tipps_zum_pm}

\cite{studie_verhalten_projektmitarbeiter}

\subsection{Aufbau und Ablauf der Arbeit}
Die vorliegende Arbeit gliedert sich in fünf Teile: Einleitung, Projektplanung, Projektdurchführung, Projektabschluss und Fazit, wobei die Kapitel zwei bis vier den Hauptteil der Arbeit ausmachen. Nach jedem Hauptteil der Arbeit wird eine Gegenüberstellung von Problem und Lösung geliefert. 
\\ \\
Im einleitenden Teil wird der Problembereich erläutert und es werden wichtige Begriffe für die weiteren Teile definiert und von einander abgegrenzt. 
\\ \\
Der Hauptteil 2: Projektplanung erläutert typische Probleme während der Planung durch verschiedene Projektbeteilige und gibt Tipps zum Umgang mit Planungsproblemen. 
\\ \\
Im Hauptteil 3: Projektdurchführung werden typische Szenarien und Probleme während der Durchführung eines Software-Projekt gezeigt. 
\\ \\
Das letzte Kapitel des Hauptteils 4: Projektabschluß beschäftigt sich mit dem erfolgreichen oder nicht-erfolgreichen Projektabschluß und gibt auch hier beispielhafte Problemsituationen und Lösungsansätze. 
\\ \\
Im Teil Fazit werden Erkenntnisse aus dem Hauptteil zusammengefast und bewertet. 
\subsection{Was ist ein Projekt?}


\begin{quote}
\colorbox{blue!5}{\textbf{Das Geheimnis des Erfolges ist die Beständigkeit des Ziels.}}\footnote{Benjamin Disraeli, britischer Schriftsteller u. Premierminister (21. Dez. 1804 - 19. April 1881)}
\end{quote}
Ein Projekt verfolgt immer ein Ziel, also eine bestimmte \textit{Zielvorgabe}. Zum Beispiel die Umsetzung eines Auftragsarbeit, oder die Erstellung einer Smartphone-Anwendung. Die Zielvorgabe besteht aus mehreren Einzelaufgaben und 

\subsection{Projektphasen}

Projektphasen lassen sich im groben in drei Bereiche einteilen: Planung, Durchführung und Abschluss. 

\subsection{Projektbeteiligte}

\pagebreak
\section{Projektphase 1: Planung}
\framebox{
	\colorbox{red!20}{
		\parbox{0.44\textwidth}{
Dolore te feugait nulla facilisi nam liber tempor cum soluta nobis eleifend option. Amet consectetuer adipiscing elit sed diam nonummy nibh euismod tincidunt ut laoreet. In iis qui facit eorum; claritatem Investigationes demonstraverunt lectores legere me. In vulputate velit esse molestie consequat vel illum dolore. Wisi enim ad minim, veniam quis nostrud exerci. Facer possim assum typi non habent claritatem insitam est usus legentis lius quod.
		}
	}
}
\framebox{
	\colorbox{green!20}{
		\parbox{0.44\textwidth}{
Dolore te feugait nulla facilisi nam liber tempor cum soluta nobis eleifend option. Amet consectetuer adipiscing elit sed diam nonummy nibh euismod tincidunt ut laoreet. In iis qui facit eorum; claritatem Investigationes demonstraverunt lectores legere me. In vulputate velit esse molestie consequat vel illum dolore. Wisi enim ad minim, veniam quis nostrud exerci. Facer possim assum typi non habent claritatem insitam est usus legentis lius quod.
		}
	}
}

\pagebreak

\pagebreak
\section{Projektphase 2: Realisierung und Steuerung} 

\pagebreak
\section{Projektphase 3: Abschluss und Ergebnis} 

%Beginn einer neuen Seite ohne Kopf- und Fußzeile
\clearpage
\thispagestyle{empty}
\bibliography{bib/bibliography}{}
\bibliographystyle{alphadin}


\end{document}
%-------------------
%Hier endet der Text deiner Hausarbeit
%-------------------