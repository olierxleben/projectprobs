\documentclass[10pt]{beamer}

\usepackage[utf8]{inputenc}
\usepackage[german]{babel}
\usepackage{amsmath}
\usepackage{amsfonts}
\usepackage{amssymb}
\usepackage{tikz}
\usepackage{enumitem}
\usepackage{listings}
\usepackage{xcolor}
\usepackage[german,lined]{algorithm2e}
\usepackage{float}

\usetheme{Rochester}
\useinnertheme{rectangles}
\useoutertheme{default}

\lstset{
	basicstyle=\small\ttfamily,
	keywordstyle=\color{blue},
	showstringspaces=true}

\title{Probleme im Projektmanagement und Führungstipps}
\author{Oliver Erxleben}
\institute{Hochschule Osnabrück}
\date{\today}

\setitemize{label=\usebeamerfont*{itemize item}%
  \usebeamercolor[fg]{itemize item}
  \usebeamertemplate{itemize item}}

\begin{document}
	\frame{\titlepage}
	
	\begin{frame}{Folie}
		\begin{itemize}
			\item Aufteilung des Projekts in unabhängige Module, Bibliotheken und Anwendungen.
			\item Parsing der Intel HEX-Dateien und Kommunikation mit dem ZPU-Treiber in jeweils eigene Bibliotheken, die dann aus Anwendungen heraus genutzt werden.
			\item Die erstellten Bibliotheken sind anwendungsunabhängig und können später auch in Drittanwendungen verwendet werden.
		\end{itemize}
	\end{frame}



\end{document}